\documentclass[12pt,a4paper]{report}
\usepackage[latin1]{inputenc}
\usepackage{amsmath}
\usepackage{amsfonts}
\usepackage{amssymb}
\usepackage{setspace}
\author{Jesus Gonzalez, Tomas McCandless}
\title{CS 388P Writeup}
\date{October 9, 2012}
\begin{document}
\maketitle
\hyphenpenalty=1000
\doublespace

jsdklshasdsf \cite{Gib99} \cite{Vlr03}

Models of parallel computation should attempt to satisfy two goals that are in conflict: a model should be sufficiently abstract to allow algorithm designers to
write programs that are simple and portable across architectures, and a model should also expose some low-level architectural details to allow for optimization. 
The PRAM model is both widely used and simple, yet it has been criticized for being too high-level and thus failing to accurately model parallel machines.
Specifically, the PRAM does not model realities of current parallel machines, such as bandwidth limitations. Similarly, network-based models such as the
hypercube have been criticized for being too low-level, failing to be widely reflect the current technological state of parallel machines. 

\singlespace
\begin{thebibliography}{9}

\bibitem[1]{Gib99} P. Gibbons, Y. Matias, V. Ramachandran. "Can a shared memory model serve as a bridging model for parallel computation?" Theory of Computing Systems Special Issue on papers from SPAA'97, vol. 32, no. 3, 1999, pp. 327-359
\bibitem[2]{Vlr03} V. Ramachandran, B.Grayson, M. Dahlin.(2003) Emulations between QSM, BSP and LogP: A framework for general-purpose parallel algorithm design. Journal of Parallel and Distributed Computing, vol. 63, 2003, pp. 1175-1192. 
\end{thebibliography}

\end{document}
